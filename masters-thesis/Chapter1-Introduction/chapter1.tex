%!TEX root = ../thesis.tex
%*******************************************************************************
%*********************************** First Chapter *****************************
%*******************************************************************************
\chapter{Introduction}

\section{Motivation}
\par
There have been several exciting developments in neural image compression
recently, and there now methods that consistently outperform classical methods
such as JPEG, WebP and BPG (\cite{toderici2017full}, \cite{theis2017lossy},
\cite{rippel2017real}, \cite{balle2018variational}, \cite{johnston2018cvpr},
\cite{mentzer2018cvpr}).

\par
The first advantage of ML-based image codecs, is that they can adapt to the
statistics of each individual image much better than even the best hand-crafted
methods. This allows them to compress images to much fewer bits, while retaining good
perceptual quality. A second advantage is that they are generally far easier to
adapt to new media formats, such as lightfield cameras, $360^\circ$ images,
Virtual Reality (VR), video streaming etc. The purposes of compression and the
devices on which the encoding and decoding is performed varies greatly, from
archiving terabytes of genetic data for later research on a supercomputer,
through compressing images to be displayed on a blog or a news article to
improve their loading time in a user's web browser, to
streaming video on a mobile device. Classical methods are usually
``one-size-fits-all'', and their compression efficiency can severly degrade when
attempting to compress media for which they were not designed. Designing good
hand-crafted codecs is difficult, can take several years, and
requires the knowledge of many experts. ML techniques on the other hand allow to
create equally or better performing, and much more flexible codecs within a few
months at significantly lower cost.

\par
The chief limitation of current neural image compression methods is while most
models these days are trained using gradient-based optmizers, quantization, a
key step in the (lossy) image compression pipeline, is an inherently
non-differentiable operation. Hence, all current methods need to resort to
``tricks'' and approximations so that the learning signal can still be passed
through the whole model. A review of these methods will be presented in Chapter 
\ref{chapter:related_works}.

\par
Our approach differs from virtually all previous methods in that we take
inspiration from information theory (\cite{rissanen1986stochastic} and
\cite{harsha2007communication}) and neural network compression
(\cite{hinton1993keeping} and \cite{havasi2018minimal}) to develop a general
lossy compression
framework that allows us to forgo the quantization step in our compression
pipeline completely. We then apply these ideas to image compression and
demonstrate that our codec using Probabilisitic Ladder Networks 
(\cite{sonderby2016train}), an extension of Variational Auto-Encoders
(\cite{kingma2013auto}), achieves close to state-of-the-art performance on the
Kodak Dataset (\cite{kodakdataset}) with no fine-tuning of our architecture.

\section{Thesis Contributions}
\par
The contributions of our thesis are as follows:
\begin{enumerate}
\item A \textbf{comparative review} of recent influential works in the field of
  neural image compression.

\item The development of a \textbf{general lossy compression
    framework} that allows to forgo the quantization step in the compression pipeline,
  thus allowing end-to-end optimization of models using of gradient-based methods.

\item A \textbf{novel image compression algorithm} using our framework, that 
  achieves close to state-of-the-art performance on the Kodak Dataset
  \cite{kodakdataset} without any fine-tuning of model hyperparameters.

\item Three \textbf{sampling algorithms} for multivariate Gaussian
  distributions, that can be readily used in our compression framework.
\end{enumerate}

\section{Thesis Outline}
\par 
Our thesis begins with an introduction to the field of neural image compression.
We first review concepts in image compression in Section
\ref{sec:intro_image_compression}, such as lossless versus lossy
compression, the rate-distortion trade-off and linear and non-linear transform
coding. We emphasize the fundamental role quantization plays in virtually all
previous approaches in image compression. In Section
\ref{sec:intro_theoretical_foundations}, we shift our focus to information
theory, where we introduce the Minimum Description Length (MDL) Principle
(\cite{rissanen1981universal}) and the Bits-back Argument
(\cite{hinton1993keeping}). Taking inspiration from these, as well as from
\cite{harsha2007communication} and \cite{havasi2018minimal}, in Section
\ref{sec:compression_without_quantization}
we develop a general framework for the lossy compression of data, and show how
it is related to quantization.

\par
In Chapter \ref{chapter:related_works}, we give a comparative review of recent
influential developments in neural image compression. We examine their whole
pipeline: the datasets used, their architectures, the ``tricks'' and
approximations used to circumvent the non-differentiability of quantization,
their coding methds, training procedures and evaluation methods.

\par
In Chapter \ref{chapter:method}, we describe our proposed method. We explain our
choice of dataset, and preprocessing steps. We give a detailed description of
our model and why we chose it. We describe our
training procedure, based on ideas from \cite{sonderby2016train},
\cite{higgins2017beta}, \cite{balle2018variational} and \cite{dai2019diagnosing}.
Next, we present 3 ``codable'' sampling techniques, that can be used in our
compression framework and point out their strengths and weaknesses.

\par
Finally, in Chapter \ref{chapter:experiments} we compare our trained models to
current compression algorithms, both classical such as JPEG and BPG, and
the current state-of-the-art neural methods (\cite{balle2018variational}). In
particular, we compare these methods by their compression rates for a given
perceptual quality as measured by the two most popular perceptual metrics, Peak
Signal-to-Noise Ratio (PSNR) (\cite{psnr}) and the Multi-scale Structural
Similarity Index (MS-SSIM) (\cite{msssim}). We achieve close to
state-of-the-art performance, with no fine-tuning of model hyperparameters.
We also present some further analysis of our chosen models, to empirically
justify their use, as well as to analyze some of the aspects that were not of
primary concern of this work, such as coding speed.
\paragraph{}

% \section{Notation and Basic Concepts}
% \paragraph{}
% It will be useful to clarify some of the notation throughout this work.
% \begin{itemize}
% \item Vectors will be denoted by boldface lowercase letters: $\vec{u}, \vec{x}, ...$
% \item Matrices will be denoted by uppercase letters: $A, M, ...$
% \item Probability mass functions will be denoted by uppercase letters: $P(x),
%   Q(z), ...$
% \item Probability density functions will be denoted by lowercase letters: $p(y),
%   q(u), ...$
% \item In general, exact/continuous values will be denoted by unannotated letters
%   (e.g. $z_i, \vec{w}$), their quantized counterparts denoted by a hat
%   ($\hat{z}_i, \hat{\vec{w}}$) and their approximate counterparts by a tilde
%   ($\tilde{z}_i, \vec{\tilde{w}}$).
% \item $\Exp_{p(x)}[f(x)]$ denotes the expected value of $f(x)$ with respect to
%   the mass / density $p(x)$, i.e.:
%   \[
%     \Exp_{p(x)}[f(x)] = \int_\Omega f(x) \d p(x),
%   \]
%   where $\Omega$ is the sample space. As $\Omega$ will usually denote $\Reals^n$
%   or will be understood from context, it will be omitted, and the integral will
%   be rewritten as
%   \[
%     \Exp_{p(x)}[f(x)] = \int f(x)p(x) \d x.
%   \]
% \item $H[X]$ denotes the Shannon entropy of the random variable $X$. If $X$ is
%   discrete, then it is defined as
%   \[
%     -\sum_{X=x}P(X=x)\log P(X=x).
%   \]
%   If it is continuous, then it will refer to the \textit{differential entropy}
%   of $X$, namely
%   \[
%     -\int_{\X}\log p(x) \d p(x),
%   \]
%   where $\X$ denotes the support of $X$.
%   \paragraph{Note:} we used the natural logarithm in our definition of entropy,
%   and hence its units are \textbf{nats}. If we used the base 2 logarithm
%   instead, the units would be \textbf{bits}.
% \item $\KL{q(x)}{p(x)}$ denotes the Kullback-Leibler divergence between two
%   distributions and is defined as
%   \[
%     \KL{q(x)}{p(x)} = \Exp_{q(x)}\left[\log\frac{q(x)}{p(x)}\right].
%   \]
% \item $I[X : Y]$ denotes the mutual information between random variables $X$ and
%   $Y$ and is defined as
%   \[
%     I[X : Y] = \KL{p(x, y)}{p(x)p(y)},
%   \]
%   where $(X, Y) \sim p(x, y)$ and $p(x)$ and $p(y)$ denote the marginals.
% \end{itemize}
\section{Image Compression}
\label{sec:intro_image_compression}
\par
The field of image compression is a vast topic that mainly spans over the fields
of computer science and signal processing, but incorporates methods and
knowledge from several
other disciplines, such as mathematics, neuroscience, psychology
and photography. In this section, we introduce the reader to the basics of the
topic, starting with source coding, then through lossy compression we arrive at
the concepts of rate and distrotion. Finally, we introduce transform coding, the
category in which our work falls as well.

\subsection{Source Coding}
From a theoretical point of view, given some source $S$, a sender and a
receiver, compression may be described as the aim of the sender communicating an
arbitrary sequence $X_1, X_2, \hdots, X_n$ taken from $S$ to the receiver in as few bits
as possible, such that the receiver may recover relevant information from the message.
If the receiver can always recover all the information from the message of the sender, we
call the algorithm \textbf{lossless}, otherwise we call it \textbf{lossy}. 
\par
At first it might not seem intuitive to allow for lossy compression, and in some
domains this is definitely true, e.g. in text compression. However, 
humans' audio-visual perception is neither completely aligned with the range of
what can be digitally represented, nor does it always scale the same way
(\cite{eskicioglu1994image}, \cite{psnr}, \cite{gupta2011modified}). Hence,
there is a great opportunity for compressing media in a lossy way by discarding
information with the change being imperceptible for a human observer, while
making significant gains in size reduction.
\subsection{Lossy Compression}
As the medium of interest in lossy compression is generally assumed to be a
real-valued vector $\vec{x} \in \Reals^N$, such as RGB pixel intensities in an
image or frequency coefficients in an audio file, the usual pipeline consists of 
an encoder $C \circ \Enc$, mapping a point $\vec{x} \in \Reals^N$ to a string of bits and a
decoder mapping from bitstrings to some reconstruction $\hat{\vec{x}}$. The
factor of the encoder $\Enc$  can be understood as a map from $\Reals^N$ to a
finite symbol set $\A$, called a \textbf{lossy encoder}, and $C$ can be
understood as a map from $\A$ to a
string of bits called a \textbf{lossless code} (\cite{goyal2001theoretical}).
We examine both $\Enc$ and $C$ in more detail in Section
\ref{sec:transform_coding}.
The decoder then can be thought of as inverting the code first and then using an
approximate inverse of
$\Enc$ to get the reconstruction $\hat{\vec{x}}$: $\Dec \circ C^{-1}$.
Given these it is of paramount importance to quantify
\begin{itemize}
\item The \textbf{distortion} of the compressor. On average, how closely does
  $\hat{\vec{x}}$ resemble $\vec{x}$?
\item The \textbf{rate} of the compressor. On average, how many bits are
  required to communicate $\vec{x}$? We want this to be as low as possible of course.
\end{itemize}

\subsection{Distortion}
\label{sec:intro_distrotion}
In order to measure ``closeness'' in the space of interest $\ImSpace$,
a distance metric $d(\cdot, \cdot): \ImSpace
\times \ImSpace \rightarrow \Reals$ is introduced. Then, the distortion $D$ is 
is defined as
\[
  D = \Exp{d(\vec{x}, \hat{\vec{x}})}{p(\hat{\vec{x}})}.
\]
A popular choice of $d$, across many domains of compression is the normalized $L_2$ metric
or Mean Squared Error (MSE), defined as
\[
  d(\vec{x}, \hat{\vec{x}}) = \frac{1}{N} \sum_{i}^N (x_i - \hat{x}_i)^2, \quad
  \ImSpace = \Reals^N.
\]
It is a popular metric as it is simple, easy to implement and has nice
interpretations in both the Bayesian (\cite{bishop2013pattern}) and the MDL
(\cite{hinton1993keeping}, to be introduced in Section \ref{sec:mdl}) settings.
In the image compression setting, however, the MSE is problematic, since 
optimizing for it does not necessarily translate to obtaining pleasant-looking
reconstructions (\cite{zhao2015loss}). Hence, more appropriate, so-called
\textit{perceptual metrics} were developed. The two most common ones used today
are Peak Signal-to-Noise Ratio (PSNR) (\cite{psnr}, \cite{gupta2011modified}) and the
Structural Similarity Index (SSIM) (\cite{wang2004image}) and its multiscale
version (MS-SSIM) (\cite{msssim}). Crucially, these two metrics are also
differentiable, thus they lend themselves for gradient-based optimization.

\subsection{Rate}
We noted above that the code used after the lossy encoder is lossless. To
further elaborate, in virtually all cases it is an \textbf{entropy code}
(\cite{goyal2001theoretical}). This means that we assume that each symbol
in the representation $\vec{z} = \Enc(\vec{x})$ has some probability mass
$P(z_i)$. A fundamental result by Shannon states that $\vec{z}$ may not be
encoded losslessly in fewer than $H[\vec{z}]$ nats:
\begin{theorem}{(Proven by \cite{shannon1998mathematical}, presented as stated in
    \cite{mackay2003information})}
$N$ i.i.d. random variables each with entropy $H[X]$ can be compressed into more
than $N\cdot H[X]$ bits with negligible risk of information loss, as $N
\rightarrow \infty$; conversely if they are compressed into fewer than $NH[X]$ bits it is virtually certain that information will be lost.
\end{theorem}
Entropy codes, such as Huffman codes (\cite{huffman1952method}) or Arithmetic
Coding (\cite{rissanen1981universal}) can get very close to this lower bound.
We discuss coding methods further in Section \ref{sec:coded_sampling}. 
\begin{framed}
  In particular, entropy codes can compress each symbol $z_i$ in $-\log P(z_i)$ nats.
\end{framed}
The rate (in nats) of the compression algorithm is defined
as the average number of nats required to code a single dimension of the input, i.e.
\[
  R = \frac{1}{N} H[\vec{z}].
\]
\subsection{Transform Coding}
\label{sec:transform_coding}
The issue with source coding is that coding $\vec{x}$ might have a lot of
dependencies across its dimensions. For images, this manifests on multiple
scales and semantic levels, e.g. a pixel being blue might indicate that most
pixels around it are blue as the scene is depicting the sky or a body of water;
a portrait of a face will also imply that eyes, a nose and mouth are probably
present, etc. Modelling and coding this dependence structure in very high
dimensions is challenging or intractable, and hence we need
to make simplifying assumptions about it to proceed.
\par
\textit{Transform coding} attempts to solve the above problem by decomposing the
encoder function $\Enc = Q \circ T$ into a so-called \textbf{analysis transform}
$T$ and a \textbf{quantizer} $Q$. The idea is to transform the input into a
domain, such that the dependencies between the dimensions are removed, and hence
they can be coded individually. The decoder inverts the steps of the encoder,
where the inverse operation of $T$ is called the \textbf{synthesis transform}
(\cite{gupta2011modified}).
\par
In \textit{linear transform coding}, $T$ is an invertible linear trasformation,
such as a discrete cosine transformation (DCT), as it is in the case of JPEG
(\cite{wallace1992jpeg}), or discrete wavelet transforms in JPEG 2000
(\cite{rabbani2002overview}). While simple, fast and elegant, linear transform
coding has the key limitation that it can only at most remove correlations (i.e.
first-order dependencies), and this can severly limit its efficiency
(\cite{balle2016endtrans}). Instead, \cite{balle2016endtrans} propose a method for
\textit{non-linear transform coding}, where $T$ is replaced by a highly
non-linear transformation, and its inverse is now replaced by an approximate
inverse, which is a separate non-linear transformation. Both $T$ and its
approximate inverse are learnt, and the authors show that with a more
complicated transformation they can easily surpass the performance of the much
more fine-tuned JPEG codecs.
\par
Our work also falls into this line of research, although with signifcant
differences, which will be pointed out later.

\subsection{The Significance of Quantization in Lossy Compression}
\par
The reason why quantization is required in lossy compression algorithms, is
because it allows to reduce the information content of data. To study the
precise meaning of this, we put the problem in a formal setting.
Let us define the quantizer as a function $[\cdot]: R \rightarrow S$, where
$R$ is the original representation space (in transform coding this would be the
image $T(\Reals^N)$), and a quantized space $S$ (usually $\Ints^N$). $[\cdot]$
is always many-to-one mapping. Let $[s]^{-1} = \{x \in R \mid [x] = s \}$ be
the preimage of $s$. Then, we have the further requirement on $[\cdot]$ 
that the fibres of $S$ partition $R$, i.e.
\[
  \text{if } s \neq t \Rightarrow [s]^{-1} \cap [t]^{-1} = \emptyset.
\]
A popular option for the quantizer is the rounding function, mapping $[\cdot]: \Reals
\rightarrow \Ints$, where for each integer $z \in \Ints$ it is defined as $x
\in \left[z - \frac12, z + \frac12\right) \mapsto z$. Given some probability mass $P(x)$ 
for some data $x$, we have seen that using entropy coding $x$ can be encoded in
$-\log P(x)$ nats. The way quantization enables better compression, is that it
aggregates the probability mass of all elements in $[s]^{-1}$ into the mass of $s$.
Namely, for each $s$, the quantizer induces a new probability mass function
$\hat{Q}(s)$, such that
\[
  \hat{Q}(s) = \int_{[s]^{-1}} p(x) \d x,
\]
where the integral is replaced by summation for discrete $[s]^{-1}$. This will allow
us to code $x$ in potentially much fewer nats. To put it precisely, assume $[x] =
s$, then
\[
  -\log \hat{Q}(s) = -\log \int_{[s]^{-1}} p(x) \d x \leq -\log P(x).
\]
This is at the cost of introducing distortion (see Section
\ref{sec:intro_distrotion}), as we will not be able to reconstruct $x$ from $s$.
In particular, quantization is vital for continuous $x$, as the probability mass of each
individual $x$ is 0, and hence we would require $-\log P(x) = \infty$ nats to
encode them without quantization.

\section{Theoretical Foundations}
\label{sec:intro_theoretical_foundations}
\par
We now shift our focus from image compression to the foundations of
neural compression. We begin with the Minimum Description Length (MDL)
Principle (\cite{rissanen1986stochastic}) and the Bits-Back Argument
(\cite{hinton1993keeping}), the two core theoretical guiding
principles of this work. We then see how based on these, as well as on
more recent work (\cite{harsha2007communication}, \cite{havasi2018minimal})
we can develop a general ML-based compression framework that does not include
quantization in its pipeline, thus allowing gradient-based optimization methods
to be used in training our compression algorithms.

\subsection{The Minimum Description Length Principle} 
\label{sec:mdl}
Our approach is based on the Minimum Description Length (MDL) Principle
(\cite{rissanen1986stochastic}). In essence, it is a formalization of Occam's
Razor, i.e. the simplest model that describes the data well is the best model of
the data (\cite{grünwald2007minimum}). Here, ``simple'' and ``well'' need to be
defined, and these definitions are precisely what the MDL principle gives us.
Informally, it asserts that given a class of hypotheses $\Hypos$ (e.g. a certain
statistical model and its parameters) and some data $\Data$, if a particular
hypothesis $H \in \Hypos$ can be described with at most $L(H)$ bits and the using the
hypothesis the data can be described with at most $L(\Data \mid H)$ bits, then the
minimum description length of the data is
\begin{equation}
\label{eq:min_desc_princ}
  L(\Data) = \min_{H \in \Hypos}\{ L(H) + L(\Data \mid H) \},
\end{equation}
and the best hypothesis is an $H$ that minimizes the above quantity.
\par
Crucially, the MDL principle can thus be interpreted as telling us that
\textbf{the best model of the data is the one that compresses it the most}.
This makes Eq \ref{eq:min_desc_princ} a very appealing learning objective for
optimization-based compression methods, ours included.
Below, we briefly review how this has been applied so far and how it translates
to our case.
\subsection{The Bits-back Argument}
Here we present the bits-back argument, introduced in
\cite{hinton1993keeping}. The main goal of their work was to develop a
regularisation technique for neural networks, and while they talk about the
compression of the model, the first method that realized bits-back
 efficiency came much later, developed by \cite{havasi2018minimal}. 
Although the argument is essentially just the direct application of the MDL
principle, it can seem quite counter-intuitive at first. Hence, we begin this
section with an example to illustrate the goal of the argument,
and only then move on to formulate it in more generality.

\paragraph{Example}
Let us be given a simple regression problem on the dataset $\Data = (\X, \Y)$,
where $\X = (x_1, \hdots, x_n), \Y = (y_1, \hdots, y_n)$ are both one
dimensional input and target sets and $(x_i, y_i)$ are a corresponding training
pair. Assume we wish to fit a simple model:
\[
  \hat{y} = f(x) = \alpha x + \beta,
\]
where we wish to learn the parameters $\alpha$ and $\beta$.
Assuming a Gaussian likelihood with mean 0 and variance 1 on the residuals $\delta
= y - \hat{y}$ ,
\[
  p(\delta \mid x, \alpha, \beta) = \Norm{\delta \mid 0, 1},
\]
a popular way of fitting the model is using Maximum Likelihood Estimation (MLE),
i.e. maximizing $\prod_i p(\delta_i \mid x_i, \alpha, \beta)$ which is equivalent to
minimizing the negative logarithm of this quantity, $-\sum_i \log p(\delta_i \mid
x_i, \alpha, \beta)$. It can be easily seen that this works out to be equivalent
to minimizing the Mean Squared Error (MSE) between the predicted values and the
targets:
\[
  L(\Data \mid \alpha, \beta) = \frac{1}{n} \sum_i (y_i - f(x_i))^2.
\]
A usual issue with MLE algorithms is that they are heavily overparameterized for
the problem they are supposed to be solving, and hence can easily overfit (this
is most likely not an issue with our toy model, but we shall pretend for the
sake of the argument). In order to solve this issue, a standard technique is to
introduce some regularisation term to the loss. Here we are intereseted in
applying the MDL principle directly.
\par
Before we discuss how it is applied, we must make precise the setting in which
it \textit{can} apply. In particular, the MDL principle assumes the form of a
communications problem. Assume two parties, Alice and Bob share $\X$, and some
other arbitrary pre-agreed information, but only Alice has access to $\Y$. Then,
the MDL principle asks for the minimal message that Alice needs to send to Bob,
such that he may recover $\Y$ completely. With this setup in mind, we can continue.
\par
In order to apply the MDL principle, we need to be able to calculate the
MDLs of the data given a hypothesis and the MDLs of our hypotheses.
Notice, that the former is in fact already available
in the form of the MSE for a given hypothesis, and hence $L(\Data \mid \alpha,
\beta)$ is not an overload of notation. In order to code the hypothesis
(the pair ($\alpha, \beta$) in our
case), we need to define two distributions over our parameters: a prior $P_\theta$, that
gives us the regularizing effect and stays fixed, and a posterior $Q_\phi$, the
distribution that we learn and assume that our parameters actually come
from it. We use the $\theta$ and $\phi$ to denote the sufficient
statistics of the prior and posterior, respectively. Now, learning changes, as
we are no longer optimizing a single hypothesis ($\alpha, \beta$), but a whole
class of hypotheses $Q_\phi(\alpha, \beta)$, by finding the best fitting set
of sufficient statistics $\phi$ for our dataset. Thus, our initial data
description length now becomes an expectation over the possible hypotheses:
\[
  L(\Data \mid \phi) = \Exp{L(\Data \mid \alpha, \beta)}{Q_\phi}.
\]
Defining the regularizing term, however, turns out to be trickier than expected,
and lies at the core of the bits-back argument. We seek to find the minimum
description length of a hypothesis $(\alpha, \beta)$. Using a $Q_\phi$, we know
we can encode a concrete hypothesis in $-\log Q_\phi (\alpha, \beta)$ nats, and thus a
reasonable first guess for the MDL would be
\begin{equation}
\label{eq:hypothesis_entropy}
  \Exp{-\log Q_\phi(\alpha, \beta)}{Q_\phi},
\end{equation}
i.e. the Shannon entropy of $Q_\phi$.
This turns out to be wrong, however, for the reason that Bob should be able to
decode Alice's message, and since he does not have access to $Q_\phi$, he cannot
do this. 
At this point, we note, that as $P_\theta$ is fixed, we may assume that Alice
and Bob share it a priori. This allows us to code a pair $(\alpha, \beta)$ in
$-\log P_\theta(\alpha, \beta)$ nats that Bob can definitely decode, and hence a
reasonable second guess for the MDL could be
\begin{equation}
\label{eq:hypothesis_cross_entropy}
\Exp{-\log P_\theta(\alpha, \beta)}{Q_\phi},
\end{equation}
i.e. the cross entropy between $Q_\phi$ and $P_\theta$, which is also the
expected length of the actual message that gets sent to communicate the
parameters. Note, that since the hypotheses are still drawn from $Q_\phi$,
the expectaion needs to be taken over it. This also turns out to be wrong,
as the bits-back argument shows that \textit{not all} of the bits used for
the message are used to code $Q_\phi$. Once the parameters are sent, Alice also
sends every residual $\delta_i$, obtained by using the parameter set sent to Bob.
\par
Once Bob has decoded $(\alpha, \beta)$ and each $\delta_i$,
he can fully recover each $y_i$ by calculating $x_i + \delta_i$. Now, since he
has access to both $\X, \Y$ and $P_\theta$, he may also fit a $Q_\psi$ to the
data, using the same learning algorithm as Alice used to fit her $Q_\phi$. The
key observation in (\cite{hinton1993keeping}) is that so long as this learning
algorithm is \textit{deterministic}, after sufficient training Bob can achieve
$\psi = \phi$, i.e. he recovers Alice's posterior distribution. This means that
Bob is able to sample the same $(\alpha, \beta)$ pair that was sent to him (e.g.
by also sharing a random seed with Alice either before their communication or
during, at at most an $\Oh(1)$ cost, which is negligible).
\begin{framed}
This must mean, that
Alice not only communicated $Q_\phi$ itself to Bob, but also the \textit{random
  bits} that were used in conjunction with $Q_\phi$ to draw the sample $(\alpha,
\beta)$.
\end{framed}
The fact that Alice has communicated both $Q_\phi$ and the random bits in a $-\log
P_\theta(\alpha, \beta)$ nat long message, means that in order to get the cost
of communicating $Q_\phi$ only, we simply need to subtract the length of the
random bits. But since $(\alpha, \beta)$ were drawn from $Q_\phi$, their length
is going to be precisely $-\log Q_\phi(\alpha, \beta)$. Hence, the expected
hypothesis description length is the expectation of this difference, namely
\[
  \Exp{-\log P_\theta (\alpha, \beta) - (-\log Q_\phi(\alpha, \beta))}{Q_\phi}
  =  \Exp{\log \frac{Q_\phi(\alpha, \beta)}{P_\theta (\alpha, \beta)}}{Q_\phi}
  = \KL{Q_\phi}{P_\theta}.
\]
Above the rightmost term is called the \textbf{Kullback-Leibler Divergence}
between $Q_\phi$ and $P_\phi$. It is defined as
\[
  \KL{Q}{P} = \sum_{x \in \Omega} Q(x)\log\frac{Q(x)}{P(x)}
\]
for probability mass functions $Q$ and $P$, where $\Omega$ denotes the sample
space, and
\[
  \KL{q}{p} = \int_\Omega q(x) \log\frac{q(x)}{p(x)} \d x
\]
for probability density functions $q$ and $p$.
\par
The fact that Bob can ``get the random bits back ''used in sampling the
hypothesis is the namesake of the argument.

\paragraph{The general argument}
We are now ready to state the general bits-back argument. Assume Alice has
trained a model for a regression problem, on a dataset $\Data = (\X, \Y)$, with
training pairs $(\vec{x}_i, \vec{y}_i)$, and shares $\X$ with Bob. Her model has
parameters $\vec{w}$, with prior $p_\theta(\vec{w})$, and uses the likelihood
function $p(\vec{y} \mid \vec{w}, \vec{x})$, both shared with Bob. Assume that
Alice has a learned posterior $q_\phi(\vec{w} \mid \Data)$ over the weights, and now
wishes to communicate the targets $\Y$ to Bob.
\par
Then, the bits-back argument states that if Alice acts according to the MDL
principle, then she can communicate $q_\phi$ to Bob in $\KL{q_\phi}{p_\theta}$
nats, as follows:
\begin{enumerate}
\item Alice draws a random sample $\hat{\vec{w}} \sim q_{\phi}(\vec{w})$. This
  represents a message of $- \log q_\phi(\hat{\vec{w}})$ nats.
\item $\hat{\vec{w}}$ is then used to calculate the residuals $\vec{r}_i$ between
  the model's output and the targets.
\item $\hat{\vec{w}}$ is coded
  using its prior $p_\theta$, and sent to Bob alongside the residuals
  $\vec{r}_i$.
  The total length of the message that contains
  the posterior information is hence $-\log p_{\theta}(\hat{\vec{w}})$.
\item Bob, decodes $\hat{\vec{w}}$ using the same prior $p_\theta$. He then
  recovers all targets $\Y$ by adding each $\vec{r}_i$ to his model's output with
  parameters set to $\hat{\vec{w}}$ upon input $\vec{x}_i$.
\item He then trains his model using the same deterministic algorithm as Alice
  did, to recover Alice's posterior $q_\phi$. Hence, the random bits that were
  used to communicate the sample must be deducted from the cost of
  communicating $q_\phi$. The cost of these bits is precisely $-\log
  q_\phi(\hat{\vec{w}})$. Taking the expectation of the difference w.r.t. $q_\phi$,
  the total cost of communicating $q_\phi$ is
  \[
    \Exp{\log q_\phi(\vec{w}) - \log p_\theta(\vec{w})}{q_\phi} = \KL{q_\phi}{p_\theta}.
  \]
\end{enumerate}

\paragraph{Caveats of the argument}
Note, that the original argument merely derives the minimum description length
for the weights $\vec{w}$, but clearly does not achieve it (as we have to send a
message whose expected length is $\Exp{-\log p_\theta(\vec{w})}{q_\phi}$). The
authors merely state that these bits can be ``recovered'', and propose that a
``free'' auxiliary message might be coded in them, but do not give any
propositions as to how sending these bits in the first place might be avoided.
Nonetheless, as the notion of bit-back efficiency has expanded in recent years,
it is customary to call any method \textit{bits-back efficient} that transmits 
some information in $\KL{q}{p}$ nats, for some posterior $q$ and prior $p$ over 
the information.

\section{Compression without Quantization}
\label{sec:compression_without_quantization}
\par
In this section, we present a general framework for lossy data compression, based on
the arguments presented above, as well as the works of
\cite{harsha2007communication} and \cite{havasi2018minimal}. 

\par
As mentioned at the end of the previous section, the bits-back argument
postulates that communicating the distribution of the parameter set of a model
may be achieved in $K = \KL{q(\vec{w})}{p(\vec{w})}$ nats, where $q$ and $p$ are the
posterior and prior over the parameters, respectively. However, they do not give
a method for achieving this, rather they show that only $K$ nats are used to
communicate the posterior in a longer message. Furthermore, the original MDL setup
also requires to send the residuals from the model output.
\par
For compression, however, we are only interested the communicating a sample and
not its distribution, though still at bits-back efficiency. 
The correct communication problem for this was formulated by
\cite{harsha2007communication}, and it is as
follows:
\begin{framed}
Let $X$ and $Y$ be two correlated random variables, with sample spaces
$\X$ and $\Y$ respectively, and with joint distribution $p(X, Y) = q(Y \mid
X)p(Y)$. Given a concrete $x \in \X$, what is the minimal
message Alice needs to send to Bob, such he can generate a sample accorting to
the distribution $q(Y \mid X = x)$?
\end{framed}
\par 
We can interpret $\X$ as the set of all data that we might wish to compress
(e.g. the set of all RGB-coded natural images, the set of all MP3 coded audio
files, etc.), and $\Y$ as the set of latent codes of the data, from which we may
obtain our lossy reconstruction. 
\par
The solution to the above problem requires essentially the same mild assumptions
the bits-back argument does, namely that Alice and Bob are allowed to
share a fixed prior $p(Y)$ on the latent codes, as well as the seed used for
their random generators. The significance of the latter assumption is that Alice
and Bob will be able to reconstruct the same sequence of random numbers. Given
these assumptions, \cite{harsha2007communication} propose a rejection sampling
algorithm to sample from $q(Y \mid X = x)$ using $p(Y)$, depicted in Algorithm
\ref{alg:harsha_rej_sampling} in the Appendix. Alice uses this algorithm to
sample $q$, but she also keeps track of the number of proposals made by the
algorithm. Once Alice's algorihtm accepts a proposal from $p$, it is sufficient
for Alice to communicate the sample's index $K$ to Bob. Bob can then obtain the
desired sample from $q$, by simply drawing $K$ samples from $p$, and since he
can generate the same $K$ samples as Alice did, the $K$th sampe he draws is
going to be an exact sample from $q$. Clearly, the communication cost of $K$ is
$\log K$ nats. \cite{harsha2007communication} then also prove the following result.
\begin{theorem}{(\cite{harsha2007communication})}
  \label{thm:bits-back_efficiency}
Let $X$ and $Y$ be random variables as given above. And let the communication
problem be set as above. Let $T[X : Y]$ denote the MDL (in nats) of a sample
$Y=y \sim q(Y \mid X=x)$. Then,
\begin{equation}
\label{eq:harsha_upper_bound}
  I[X : Y] \leq T[X : Y] \leq I[X : Y] + 2\log \left[ I[X : Y] + 1 \right] + \Oh(1),
\end{equation}
where $I[ X : Y ]$ is called the mutual information between $X$ and $Y$, and is
defined as
\[
  I[ X : Y ] = \Exp{\KL{q(Y \mid X)}{p(Y)}}{p(X)}
\]
Furthermore, $\log K$, given by Algorithm \ref{alg:harsha_rej_sampling},
achieves the upper bound in Eq \ref{eq:harsha_upper_bound}.
\end{theorem}
\par
The above theorem tells us that while in the classical sense bits-back efficiency
is the best that we can do, it also tells us that we can get very close to it.
Hence, from now on, we shall refer to any algorithm that achieves this tight
upper bound as bits-back efficient as well.
\par
To translate this to a general ML-based compression framework, we shall switch
to notation more common in statistical modelling, concretely, we shall denote
our data by $\vec{x}$ and the latent code $\vec{z}$. Now, let us assume a
generative model over these variables, $p(\vec{x}, \vec{z}) = p(\vec{x} \mid
\vec{z})p_\theta(\vec{z})$, where $p(\vec{x} \mid \vec{z})$ is the data likelihood, and
$p_\theta(\vec{z})$ is the prior over the latent code, with sufficient
statistics $\theta$. Let us also assume an approximate posterior $q_\phi(\vec{z}
\mid \vec{x})$ over the latent code, with sufficient statistics $\phi$. Then our
framework is as follows:
\begin{enumerate}
\item Given some dataset $\Data = \{\vec{x}_1, \hdots, \vec{x}_n\}$ where the
  training examples are distributed according to $p(\vec{x})$, we fit our
  generative model to it, by fitting $\theta$ and $\phi$ using the (weighted) MDL
  objective:
  \begin{equation}
    \label{eq:mdl_elbo}
    \begin{gathered}
      L(\Data) = \Exp{L(\vec{x})}{p(\vec{x})}, \\
      \text{where } L(\vec{x}) = \Exp{L(\vec{x} \mid \vec{z}) + L(\vec{z})}{q_\phi}
      =  -\Exp{\log p(\vec{x} \mid \vec{z})}{q_\phi} + \beta\KL{q_\phi}{p_\theta}.
    \end{gathered}
  \end{equation}
  This training objective is well known in the neural generative modelling
  literature as the Evidence Lower Bound (ELBO) (\cite{kingma2013auto},
  \cite{higgins2017beta}). The expectation over $p(\vec{x})$ is usually taken over
  randomly drawn mini-batches from $\Data$ using Stochastic Gradient Descent (SGD).
  Here $\beta$ is a
  hyperparameter that can be set to trade off a smaller description length at
  the cost of worse reconstruction, or the other way around, thus allowing the
  user to reach different points on the rate distortion curve. See Section
  \ref{sec:derive_weighted_elbo} for the derivation of Eq \ref{eq:mdl_elbo} and
  discussion on its validity.

  \item Once $\theta$ and $\phi$ have been learned, we fix them (equivalent to
    sharing them with Bob in the communication problem).

  \item Now, if we wish to compress some new data $\vec{x}'$, use a bits-back
    efficient sampling algorithm (such as Algorithm
    \ref{alg:harsha_rej_sampling}) to sample $q(\vec{z} \mid \vec{x}')$ using
    $p(\vec{z})$, and use the code output of the sampling algorithm as the
    compression code, along with the random seed that was used to obtain the sample. 
    We shall refer to such algorithms as \textbf{coded sampling algorithms}.

  \item To decompress, since we always have access to the fixed prior
    $p_\theta$, and we have the random seed the compressing party used, we may
    run the coded sampling algorithm in ``decode'' mode to recover the sample $\vec{z}'$
    from $q_\phi$. Finally, we may run the reconstruction transformation of our
    generative model to recover a lossy reconstruction $\hat{\vec{x}}'$.
\end{enumerate}

\par 
This framework is inspired by the work of \cite{havasi2018minimal}, where they
used a very similar framework to achieve state-of-the-art weight compression in
Bayesian Neural Networks.
\par
In this thesis, we use this framework to train $\beta$-VAEs as our choice of
generative models, and demonstrate the efficiency of our method compared to the
state-of-the-art in neural compression. More details on this will be given in
Chapter \ref{chapter:method}. 

\subsection{Relation of Quantization to Our Framework}
\par
We present a similar argument to the one given in \cite{havasi2018minimal}.
Recall the original representation space $R$ and quantized space $S$ of a
quantizer $[\cdot]$. Recall also the Kronecker delta function on $x$, defined as 
\[
  \delta_{x}(y) = 
  \begin{cases}
    1 & \text{if } y = x \\
    0 & \text{otherwise}.
  \end{cases}
\]
Given a particular $x \in R$, we have seen that
quantization allows us to code it in $-\log \hat{Q}([x])$ nats. If we manipulate
this term slightly, we get
\begin{align*}
  -\log \hat{Q}([x]) &= \sum_{s \in S} \left[ -\delta_{[x]}(s)\log \hat{q}([x]) + \underbrace{\delta_{[x]}(s) \log \delta_{[x]}(s)}_{= 0} \right] \\
                     &= \sum_{s \in S} \delta_{[x]}(s)\log\frac{\delta_{[x]}(s)}{\hat{q}([x])} \\
                     &= \KL{\delta_{[x]}}{\hat{Q}}.
\end{align*}
This shows that quantization of a deterministic parameter set is also
bits-back efficient, with the posterior distribution family restricted to point masses. Thus
the clear advantage of our framework comes from the fact that we allow much more
posteriors than point masses. 

\subsection{Derivation of the Training Objective}
\label{sec:derive_weighted_elbo}
\par
In this section, we present the derivation of Eq \ref{eq:mdl_elbo}. Thus, let our
likelihood $p(\vec{x} \mid \vec{z})$, our latent prior $p_\theta(\vec{z})$ and
approximate posterior $q_\phi(\vec{z} \mid \vec{x})$ be given. Then, given a budget
of $C$ nats, we want to optimize the following constrained objective on the
description lengths:
\[
  \Exp{\Exp{-L(\vec{x} \mid \vec{z})}{q_\phi(\vec{z})}}{p(\vec{x})}
  \quad \text{subject to } \Exp{L(\vec{z})}{p(\vec{x})} < C.
\]
As we have seen in the sections above, these quantities can be replaced by
\begin{equation}
\label{eq:framework_hard_train_target}
\Exp{\Exp{\log p(\vec{x} \mid \vec{z})}{q_\phi(\vec{z})}}{p(\vec{x})}
\quad \text{subject to } \Exp{\KL{q_{\phi}(\vec{z} \mid
    \vec{x})}{p_{\theta}(\vec{z})}}{p(\vec{x})} < C.
\end{equation}

As we want to use gradient-based optimization of our models, we need to find a
continuous relaxation of Eq \ref{eq:framework_hard_train_target}.
To this end, we rewrite the terms inside the ``outer'' expectation as their
Lagranagian relaxation under the KKT conditions (\cite{karush2014minima},
\cite{kuhn2014nonlinear}, \cite{higgins2017beta}) and get:
\[
  \F(\theta, \phi, \beta, \vec{x}) = 
  \Exp{\log p(\vec{x} \mid \vec{z})}{q_\phi(\vec{z})}
  - \beta (\KL{q_{\phi}(\vec{z} \mid \vec{x})}{p_{\theta}(\vec{z})} - C).
\]
By the KKT conditions if $C \geq 0$ then $\beta \geq 0$, hence discarding the last
term in the above equation will provide a lower bound for it:
\[
  \F(\theta, \phi, \beta, \vec{x}) \geq
  \L(\theta, \phi, \beta, \vec{x}) =
  \Exp{\log p(\vec{x} \mid \vec{z})}{q_\phi(\vec{z})} - \beta
  \KL{q_{\phi}(\vec{z} \mid \vec{x})}{p_{\theta}(\vec{z})}.
\]
Finally, taking the expectation over this again gives
\begin{equation}
\label{eq:framework_train_target}
\begin{aligned}
  \Exp{\L(\theta, \phi, \beta, \vec{x})}{p(\vec{x})} &=
  \Exp{\Exp{\log p(\vec{x} \mid \vec{z})}{q_\phi(\vec{z})}}{p(\vec{x})} - \beta
  \Exp{\KL{q_{\phi}(\vec{z} \mid \vec{x})}{p_{\theta}(\vec{z})}}{p(\vec{x})} \\
  &= \Exp{\Exp{\log p(\vec{x} \mid \vec{z})}{q_\phi(\vec{z})}}{p(\vec{x})} - \beta
  I[ \vec{x} : \vec{z} ] \\
  &= \Exp{L(\vec{x})}{p(\vec{x})}.
\end{aligned}
\end{equation}
This is the training objective of $\beta$-VAEs first derived in
\cite{higgins2017beta}, although we note that it is applicable any generative
model where the assumed conditions are present.
\par
An important caveat of the above formulation is that the samples from
$p(\vec{x})$ should comparable, in the sense the initial hard optimization objective
of setting an average nat budget is reasonable. In the case of image data, if
all images are the same size, this is fine, as our continuous relaxation will
allow for images with high information content to have slightly longer code
lenghts than $C$ nats and ones with low information content will have shorter
lengths. However, if we used different sized images during training, it would be
less justified to set the same average code budget for, say, a $200 \times 300$ pixel
image and a $2000 \times 2000$ image, as the latter will naturally contain more
information than the former. Hence in this case it would be more reasonable
to make the budget a function of the number of pixels the image contains,
although this might make the formulation of the training objective much harder.
\par
An approach taken in all neural image compression methods we examined is instead
to train on random, but equal-sized patches extracted from each training image.
While other works do this to make training more computationally feasible. As
far as we are aware, we are the first ones to argue that this practise is not only
convenient, but mandatory for the training procedure to be sound.

\nomenclature[z-VAE]{VAE}{Variational Auto-Encoder}
\nomenclature[z-MSE]{MSE}{Mean Squared Error}
\nomenclature[z-MAE]{MAE}{Mean Absolute Error}
\nomenclature[z-PLN]{PLN}{Probabilistic Ladder Network}
