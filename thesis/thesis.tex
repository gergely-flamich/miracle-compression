
% ==============================================================================
% ==============================================================================
%
% HEADER
%
% ==============================================================================
% ==============================================================================

\documentclass{article}

\usepackage[T1]{fontenc}
\usepackage[utf8]{inputenc}
\usepackage[english]{babel}
\usepackage{graphicx}

\usepackage[
backend=biber,
style=alphabetic,
%citestyle=authoryear
]{biblatex}

\usepackage{amssymb}
\usepackage{amsmath}
\usepackage{graphicx}
\usepackage{float}
\usepackage{hyperref}
\usepackage{algorithm, algpseudocode}


\usepackage[a4paper, total={6in, 8in}]{geometry}


% ==============================================================================
% ==============================================================================
%
% DEFINITIONS
%
% ==============================================================================
% ==============================================================================
\addbibresource{cite.bib}


\renewcommand{\vec}[1]{\mathbf{#1}}
\renewcommand{\d}{\,\text{d}}

\renewcommand{\L}{\mathcal{L}}

\newcommand{\A}{\mathcal{A}}
\newcommand{\E}{\mathcal{E}}
\newcommand{\F}{\mathcal{F}}

\newcommand{\Oh}{\mathcal{O}}
\newcommand{\KL}[2]{\mathrm{KL}[\,#1\,\,||\,\,#2\,]}
\newcommand{\Exp}{\mathbb{E}}
\newcommand{\I}{\mathbb{I}}

\newcommand{\Hypos}{\mathcal{H}}
\newcommand{\Data}{\mathcal{D}}

\newcommand{\ImSpace}{\mathcal{X}}

\newcommand{\Reals}{\mathbb{R}}
\newcommand{\Ints}{\mathbb{Z}}
\newcommand{\Nats}{\mathbb{N}}

\DeclareMathOperator*{\argmax}{arg\,max}
\DeclareMathOperator*{\argmin}{arg\,min}


\title{Compression without Quantization}
\author{Gergely Flamich}

% ==============================================================================
% ==============================================================================
%
% START OF THESIS
%
% ==============================================================================
% ==============================================================================

\begin{document}

\input{titlepage.tex}

% ==============================================================================
%
% DECLARATION
%
% ==============================================================================

\vspace{2cm}

\begin{center}
\Huge
\textbf{Declaration}
\end{center}

\vspace{1cm}

\large
\noindent I, Gergely Flamich of St John's College, being a candidate for the
MPhil in Machine Learning and Machine Intelligence, hereby declare that this
report and the work described in it are my own work, unaided except as may be
specified below, and that the report does not contain material that has
already been used to any substantial extent for a comparable purpose.


\vspace{2cm}

\large
\noindent
Wordcount: \textbf{16384} words

\newpage

% ==============================================================================
%
% ACKNOWLEDGEMENTS
%
% ==============================================================================

\vspace{2cm}

\begin{center}
\Huge
\textbf{Acknowledgements}
\end{center}

\vspace{1cm}



\newpage

% ==============================================================================
%
% ABSTRACT
%
% ==============================================================================

\begin{abstract}
  We provide an implementation of our proposed method written in \texttt{Tensorflow}
  \cite{tensorflow2015-whitepaper} and \texttt{Sonnet} \cite{sonnetblog},
  available on GitHub \footnotemark.
\end{abstract}

\footnotetext{https://github.com/gergely-flamich/miracle-compression}

\newpage

\tableofcontents

\newpage

% ==============================================================================
% ==============================================================================
%
% START OF CONTENTS
%
% ==============================================================================
% ==============================================================================

\section{Introduction}
\subsection{Motivation}
- adaptive
- need for compression in a lot of new areas for which existing techniques might
not work well
lightfield cameras, 360 images / video, VR, video streams
- good handcrafted codecs are hard to design
- put the problem in a well-grounded mathematical framework, get theoretical guarantees

\subsubsection{Why Image Compression?}
\paragraph{}
Well studied problem, good literature availability, on handcrafted methods, NN
based methods and performance evaluation.

\subsubsection{Why Lossy?}
\paragraph{}
In lossless image compression we are limited by the true distribution of image
pixels in how much we can compress stuff.

In lossy compression we can make a huge saving, by only concentrating on details
that are perceptually important to the viewer.

\subsection{Our Goals}

Several metrics to optimise for:
- compression quality
- compression size
- compression time
- compressor size
- compressor power consumption
- robustness of compressor (i.e. resistance to errors / adversarial attacks)
- security / privacy of compression
- scalability: image size, image quality

\subsection{Our Contributions}
\paragraph{}

We present:
\begin{itemize}
\item a brief introduction to the technical background on neural-network
  based lossy image compression
\item a review of recent influential works in the area

\item a novel image compression algorithm trained on the CLIC 2018 dataset \cite{clic2018}

\item experiments and analysis of performance to confirm its theoretical properties
\end{itemize}

\subsection{Thesis Outline}
\paragraph{}

\section{Background}
\subsection{Notation and Basic Concepts}
\paragraph{}
It will be useful to clarify some of the notation throughout this work.
\begin{itemize}
\item Vectors will be denoted by boldface lowercase letters: $\vec{u}, \vec{x}, ...$
\item Matrices will be denoted by uppercase letters: $A, M, ...$
\item Probability mass functions will be denoted by uppercase letters: $P(x),
  Q(z), ...$
\item Probability density functions will be denoted by lowercase letters: $p(y),
  q(u), ...$
\item $\Exp_{p(x)}[f(x)]$ denotes the expected value of $f(x)$ with respect to
  the mass / density $p(x)$, i.e.:
  \[
    \Exp_{p(x)}[f(x)] = \int_\Omega f(x) \d p(x),
  \]
  where $\Omega$ is the sample space. As $\Omega$ will usually denote $\Reals^n$
  or will be understood from context, it will be omitted, and the integral will
  be rewritten as
  \[
    \Exp_{p(x)}[f(x)] = \int f(x)p(x) \d x.
  \]
\item $\KL{q(x)}{p(x)}$ denotes the Kullback-Leibler divergence between two
  distributions and is defined as
  \[
    \KL{q(x)}{p(x)} = \Exp_{q(x)}\left[\log\frac{q(x)}{p(x)}\right].
  \]
\item $I[X : Y]$ denotes the mutual information between random variables $X$ and
  $Y$ and is defined as
  \[
    I[X : Y] = \KL{p(x, y)}{p(x)p(y)},
  \]
  where $(X, Y) \sim p(x, y)$ and $p(x)$ and $p(y)$ denote the marginals.
\end{itemize}
\subsection{Image Compression}
Main goal: use the statistical / topological properties of images and properties
of the human visual system (HVS) to achieve
better compression rates than generic data compression algorithms

Types of image compression: lossless and lossy
\cite{townsend2019practical}

lossy is suited for \textbf{natural images} -> imperceptible loss of quality can
lead to dramatic reduction in bit-rate

rate-distorsion
pnsr
\paragraph{}
The general setup for lossy image compression has two key, competing components:
the \textbf{distorsion} and the \textbf{rate} of the compressor. Given a space
of images $\ImSpace$, we select a distance metric $d(\cdot, \cdot): \ImSpace
\times \ImSpace \rightarrow \Reals^+$. Then, the distorsion of an image $\vec{x}
\in \ImSpace$ is defined as $d(\vec{x}, \hat{\vec{x}})$, where $\hat{\vec{x}}$
is the reconstruction of the image by the compression algorithm. The rate $R$
is the number of bits required to code $\vec{x}$.
\paragraph{}
the main goal of lossy image compression is to minimize distorsion while
maintaining a low rate. 
\subsection{Metrics}
\paragraph{}
\subsubsection{PSNR}
\cite{psnr}
\subsubsection{MS-SSIM}
\cite{msssim}
\subsection{The MDL principle and the Bits-Back Argument}
\paragraph{MDL Principle} 
Our approach is based on the Minimum Description Length (MDL) Principle
\cite{rissanen1986stochastic}. In essence, it is a formalization of Occam's
Razor, i.e. the simplest model that describes the data well is the best model of
the data \cite{grünwald2007minimum}. Here, ``simple'' and ``well'' need to be
defined, and these definitions are precisely what the MDL principle gives us.
Informally, it asserts that given a class of hypotheses $\Hypos$ (e.g. a certain
statistical model and its parameters) and some data $\Data$, if a particular
hypothesis $H \in \Hypos$ can be described with at most $L(H)$ bits and the using the
hypothesis the data can be described with at most $L(\Data \mid H)$ bits, then the
minimum description length of the data is
\begin{equation}
\label{eq:min_desc_princ}
  L = \min_{H \in \Hypos}\{ L(H) + L(\Data \mid H) \},
\end{equation}
and the best hypothesis is the $H$ that minimizes the above quantity.
\par
Crucially, the MDL principle can thus be interpreted as telling us that
\textbf{the best model of the data is the one that compresses it the most}.
This makes Eq \ref{eq:min_desc_princ} a very appealing learning objective for
optimization-based compression methods, ours included.
Below, we briefly review how this has been applied so far and how it translates
to our case.
\paragraph{Bits-Back Argument}
First, we begin with the bits-back argument, introduced in
\cite{hinton1993keeping}, which is a direct application of the above. The main
goal of this work was to develop a regularisation technique for neural networks
by framing the training of a neural network as a communication problem, where
the training input and the fixed network architecture is public, but the weights,
the network's output given a particular input, and the training targets are
only available to the sender, and the task is to communicate the
\textit{training targets} with minimal bits.
\par
Concretely, they train a Bayesian Neural Network, by equipping the weights
$\vec{w}$ with a prior $p_\theta(\vec{w})$ and a posterior $q_\phi(\vec{w})$
(parameterized by $\theta$ and $\phi$, respectively) and maximize
the \textit{evidence lower bound} (ELBO) given a likelihood $p(\Data \mid \vec{w})$:
\begin{equation}
  \label{eq:elbo_target}
  \Exp_{q_\phi}[\log p(\Data \mid \vec{w})] - \KL{q_{\phi}}{p_{\theta}}.
\end{equation}
Given a sufficiently finely quantized likelihood, the minimum description length
of the data given this model is $\Exp_{q_\phi}[-\log p(\Data \mid \vec{w})]$
\cite{shannon1998mathematical}, and hence the first term in Eq
\ref{eq:elbo_target} corresponds to $-L(\Data \mid H)$. In their work then,
\cite{hinton1993keeping} show that the second term in Eq \ref{eq:elbo_target}
is equal to $-L(H)$, which establishes a link between the variational training
objective of the BNN and the MDL principle.
\par
To do this, the encoder
\begin{enumerate}
\item trains the neural network, optimising Eq \ref{eq:elbo_target}.
\item draws a random sample $\vec{\hat{w}} \sim q_{\phi}(\vec{w})$. This
  represents a message of $\Exp_{q_\theta}[- \log q_\phi]$ nats.
\item $\vec{\hat{w}}$ is then used to calculate the residuals $\vec{r}$ between
  the network's output and the targets.
\item $\vec{r}$ is coded with $\vec{\hat{w}}$ and then $\vec{\hat{w}}$ is coded
  using its prior $p_\theta$. The total length of the message is hence $\Exp_{q_\phi}[-\log
  p(\Data \mid \vec{\hat{w}})] + \Exp_{q_\phi}[-\log p_{\theta}]$.
\end{enumerate}
Once everything has been communicated, the decoder can recover the true training
targets, but then they can also run the same training algorithm that the encoder
used to then recover the posterior $q_\phi$. This means that the code of
$\vec{\hat{w}}$ is ``free bits'' in the sense that the decoder can recover them
exactly given what they already have. 
Hence, the whole cost of drawing $\vec{\hat{w}}$ should be
subtracted from the original cost, yielding
$\Exp_{q_\phi}[-\log p_{\theta}] - \Exp_{q_\phi}[-\log q_{\phi}] =
\KL{q_{\phi}}{p_\theta}$ nats. This ``recovery'' is the namesake for the
bits-back argument.
\paragraph{MIRACLE}
Inspired by the above idea, \cite{havasi2018minimal} asked a natural question:
\textit{is it possible to communicate only the weights of a network at
  bits-back efficiency?}
\par
If the above were true, it would give a method for compressing neural networks
rather efficiently. It is clear that the coding must be different than it was in
\cite{hinton1993keeping}, as their method focused on the regularisation aspect
of the KL-divergence and is very inefficient for actual communication of the
model parameters.
\par
A second, important question that arises in conjunction with the first, natural
for compression algorithms:
\textit{is it possible trade off accuracy of a fixed neural network architecture
  for better compression rates, and vice versa?}
\par
Luckily, the answer to both of the above questions is yes, and we shall begin by
addressing the latter first. Fix a network architecture, and some data
likelihood given a weight set $p(\Data \mid \vec{\hat{w}})$. Akin to
\cite{hinton1993keeping}, we will actually train a BNN with weight prior
$p(\vec{w})$ and posterior $q(\vec{w})$. Then, given a budget of $C$ nats, we
hope to maximize the following constrained objective:
\begin{equation}
\label{eq:miracle_hard_train_target}
\Exp_{q_\phi}[\log p(\Data \mid \vec{w})] \quad \text{subject to }
\KL{q_{\phi}}{p_{\theta}} < C.
\end{equation}
We can rewrite Eq \ref{eq:miracle_hard_train_target} as its Lagranagian
relaxation under the KKT conditions \cite{karush2014minima},
\cite{kuhn2014nonlinear}, \cite{higgins2017beta} and get:
\[
  \F(\theta, \phi, \beta, \Data, \vec{\hat{w}}) = 
  \Exp_{q_\phi}[\log p(\Data \mid \vec{\hat{w}})] - \beta (\KL{q_{\phi}}{p_{\theta}} - C).
\]
By the KKT conditions if $C \geq 0$ then $\beta \geq 0$, hence discarding the last
term in the above equation will provide a lower bound for it:
\begin{equation}
\label{eq:miracle_train_target}
\F(\theta, \phi, \beta, \Data, \vec{\hat{w}}) \geq
\L(\theta, \phi, \beta \Data, \vec{\hat{w}}) =
\Exp_{q_\phi}[\log p(\Data \mid \vec{\hat{w}})] - \beta \KL{q_{\phi}}{p_{\theta}}.
\end{equation}
Notice, that this is the same as Eq \ref{eq:elbo_target}, but with the addition of
the parameter $\beta$ that will control the regularisation term and eventually
the compression cost of the weights. It is also intimately related to the
training target of $\beta$-VAEs \cite{higgins2017beta}, except for where they
regularise the distributions of activations on a stochastic layer, here the
regularisation is for the distributions of weights.
\par
Now, to answer the first question, we first need to establish the right setting
for the task, which will be another communications problem.
Concretely, given a dataset $\Data$ sampled from a distribution $p(D)$, and
$q_\phi(\vec{w})$, our trained weight posterior for a given $\beta$, what are
the bounds on the minimum description length for the posterior $L(q_{\phi})$?
\par
Under some mild assumptions, it can be shown \cite{harsha2007communication} that
in fact
\[
  \Exp_{p(D)}[L(q_{\phi})] \geq \Exp_{p(D)}[\KL{q_{\phi}}{p_{\theta}}],
\]
i.e. in this probabilistic setting bits-back efficiency is the best we can hope
for. Now, if we make the further assumption that the sender and the receiver are
allowed to \textit{share a source of randomness} (e.g. a random number generator
and a seed for it), then a rather tight upper bound can also be derived, also
due to \cite{harsha2007communication}:
\begin{equation}
\label{eq:miracle_ub}
  \Exp_{p(D)}[L(q_{\phi})] \leq I[\Data : \vec{w}] + 2 \log \left( I[\Data :
    \vec{w}] + 1 \right) + \Oh(1)
\end{equation}
where $I[D : \vec{w}] = \Exp_{p(D)}[\KL{q_{\phi}}{p_{\theta}}]$ is the
mutual information between the distribution of datasets and the weights.
\par
Eq \ref{eq:miracle_ub} is proven by exposing an algorithm that achieves the
postualted coding efficiency, an adaptive rejection sampling algorithm, which we
detail in the Appendix A. This turns out to be infeasible in the case of
MIRACLE, and instead the authors propose an importance sampling-based
approximate sampling algorithm, which is also discussed in further detail in
Appendix A. They are important, as we have used both in our project.

\paragraph{Our method} Our project is based upon the simple observation, that
the MIRACLE framework may be utilised for compression of any data where in our
model a public prior distribution $p_{\theta}$ and a learned $q_{\phi}$ is
available and we are allowed to share a source of randomness. We have already
noted the extreme similarity of the original BNN training objective to that of
the $\beta$-VAE \cite{higgins2017beta}, and indeed they are precisely the model
that we shall use for the compression of images in our case.

\subsection{Neural Networks for Image Compression}
\paragraph{}

Why not compress images straight away?
- intractable due to high dimensionality
- ineffective due to spatial and cross-channel dependencies

solution:
- map into smaller dimensional representation and reduce dependencies between dimensions

\section{Related Work}
\paragraph{}
There have been several recent advances in neural network-based compression
techniques, most notably \cite{balle2016end}, \cite{theis2017lossy},
\cite{rippel2017real}, \cite{balle2018variational}, \cite{johnston2018cvpr},
\cite{mentzer2018cvpr} 
\paragraph{}
Notably, all previous VAE-based approaches have addressed the
non-differentiablility of quantization indirectly.
\paragraph{}
\cite{theis2017lossy} use an approximation for the derivative of the rounding
operator and optimize an upper bound on the error term introduced by the
quantiztion.
\paragraph{}
In \cite{balle2016end},\cite{balle2018variational} they model the quantizer by
adding uniform noise to the samples 
\subsection{VAE-based image compression}
\section{Method}

\subsection{Architecture}
\subsubsection{VAEs}
\paragraph{}
Unlike earlier work, since we do not require the quantization step at all, we
can train a classical VAE. On the data

\subsubsection{Hierarchical VAE}
\paragraph{}

\subsubsection{Probabilistic Ladder Network}
\paragraph{}

\subsubsection{Two-Stage VAE}
\paragraph{}
\cite{dai2019diagnosing}

\subsection{MIRACLE Coding}
\paragraph{}
\cite{havasi2018minimal}

\subsubsection{Rejection sampling}
\paragraph{}
\cite{harsha2007communication}

\subsection{Arithmetic Coding}
\paragraph{}
ac reference \cite{rissanen1981universal}

\section{Experimental Results}

\cite{zhao2015loss}
\section{Discussion}
\section{Conclusion}

\printbibliography
\section*{Appendix A: Sampling algorithms}
\section*{Appendix B: Images}
\end{document}